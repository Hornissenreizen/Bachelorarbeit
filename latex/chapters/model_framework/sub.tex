\documentclass[../../main.tex]{subfiles}

\begin{document}
\section{Model Framework}
    We are interested in models with asymptotically power-law decay of the mutual information measure with respect to the distance between the tokens in the sequence. So far so good. But what does it \emph{actually} mean?

    The tokens, represented by random variables $X_t$, are elements of a finite alphabet $\Sigma$. The distance between $X_t$ and $X_{t + \tau}$ is $\tau$, and for every $t$ and every $\tau$ we want to bound
    \[
        I(X_t, X_{t + \tau}) \in \Omega(\tau^{-\alpha}), \ I(X_t, X_{t + \tau}) \in \mathcal{O}(\tau^{-\beta}) \quad ,
    \]
    for some fixed $\alpha, \beta \in \mathbb{R}_{>0}$. The first condition is the important one, while the latter ensures that $I(X_t, X_{t + \tau}) \xrightarrow{\tau \to \infty} 0$. We also may replace the latter condition by this one.

    This was straight forward. The challenging part is to define what a  model is. In the case of Markov chains this seems trivial: We define a finite set of parameters (the transition probabilities), and we get a model over $\Sigma^*$, that is for every $n \in \mathbb{N}$ the model defines a probability measure over $\Sigma^n$. Thus, the first conclusion is every model $S$ must define a probability measure over $\Sigma^n$ for every $n\in \mathbb{N}$.

    As a first formalization, $S$ is a function $S: (n, w) \mapsto [0, 1]$, for $n \in \mathbb{N}, \ w \in \Sigma^n$ s.t. $\sum_{w \in \Sigma^n} S(n, w) = 1$.

    But really, we want to restrain $S$ in order to have reasonable time and space complexity, and to ensure the model is \emph{reasonable}, which means that the language of $S_n(w)$ should look \emph{similar} to $S_{n + d}(w)$, whatever this might mean, where we used the notation $S_n(w) \equiv S(n, w)$. We also write $w_i$ for $X_i$. Really, $w$ is a 1-indexed String of $X_i$.  

    We present one strict definition for this \emph{similarity} in the following definition:

    \begin{definition}
        We say $S$ is \emph{well behaved} iff for every $n \in \mathbb{N}, \ w \in \Sigma^{n + 1}$ it holds true that
        \[
            \sum_{w_{n + 1} \in \Sigma} S_{n + 1}(w) = S_n(w_{-(n + 1)}) \quad .
        \]
    \end{definition}

    Now, we want to look at how we might restrict our model $(S_n)_{n \in \mathbb{N}} \equiv S$. One approach might be to define a model structure for every $n \in \mathbb{N}$ with parameters $\bm{\theta}$, thus $S_n \in \{ S_n(\bm{\theta}) : \bm{\theta} \in \Theta_n \}$. We write $S_{n, \bm{\theta}}$ for $S_n$ with parameters $\bm{\theta}$.
\end{document}