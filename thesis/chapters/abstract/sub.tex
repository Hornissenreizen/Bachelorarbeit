\documentclass[../../main.tex]{subfiles}

\begin{document}
\section*{Kurzfassung}
Empirische Analyse natürlicher Sprache hat ergeben, dass die Abnahme der Abhängigkeiten von Textfragmenten mit der Distanz einer bestimmten Funktion folgt. Motiviert durch diese Erkenntnis widmet sich diese Arbeit der Formalisierung und Analyse von Modellklassen, ob sie ebendieses \textbf{power-law Verhaltens} fähig sind.

Um präzise Aussage treffen zu können, bedarf es einer mathematischen Definition eines Modells und \textbf{Modellklassen}, sowie einer Formalisierung der power-law Eigenschaft. Dabei sollten die Definitionen konsistent mit gegenwärtigen Ergebnissen, die zu diesem Themengebiet vorliegen, sein. Zu diesem Zweck dient unser \textbf{Framework}.

Wir reproduzieren das etablierte Ergebnis der Unfähigkeit eines power-law Verhaltens von \textbf{Hidden Markov Models} im Kontext unseres Frameworks. Unser formales Vorgehen, sowie die Allgemeingültigkeit unserer Aussage, stellen dabei ein fortschrittliches Resultat dar.

Außerdem stellen wir ein positiv Beispiel für ein Modell mit power-law Verhalten vor, was in der Form und in dieser Formalisierung ein neues Ergebnis darstellt.

\bigskip
\subsection*{Keywords}
natural language, power-law behavior, mutual information, hidden Markov models


\end{document}